
  \begin{figure*}[t]
  \centering
  % \scalebox{0.7}
  % {
  % \begin{tikzpicture}[node distance=3cm,auto,scale=0.1,-{Latex[length=3mm,width=2mm]},>=Latex]
  \begin{tikzpicture}[node distance=3cm,auto,scale=0.5,transform shape,
-{Latex[length=1.8mm,width=1mm]},>=Latex]

  % \begin{tikzpicture}[node distance=3cm,auto,scale=0.6, every node/.style={scale=0.6},->]
    % \tikzstyle{state}=[circle,draw=black,minimum size=13mm]
    % \tikzstyle{sink}=[draw=black,minimum size=12mm]
    \tikzset{state/.style={circle,draw=black,minimum size=14mm},
          sink/.style={draw=black,minimum size=12mm}}

    \node[state](n){$n$};
    \node[state,right of=n](n1){$n-1$};
    \node[state,right of=n1](n2){$n-2$};
    \node[right of=n2](e1){\Huge\textbf{...}};
    \node[state,right of=e1](nn2){$2$};
    \node[state,right of=nn2](nn1){$1$};
    \node[state,right of=nn1](nn0'){$0'$};
    \node[sink,right of=nn0'](s1){$\Tilde{1}$};
    
    \node[state,below of=n](n'){$n'$};
    \node[state,right of=n'](n1'){$(n-1)'$};
    \node[state,right of=n1'](n2'){$(n-2)'$};
    \node[right of=n2'](e1'){\Huge\textbf{...}};
    \node[state,right of=e1'](nn2'){$2'$};
    \node[state,right of=nn2'](nn1'){$1'$};
    \node[sink,right of=nn1'](s0){$\Tilde{0}$};
    

    \draw(n) edge [dashed, red] (n1);
    \draw(n1) edge [dashed, red] (n2);
    \draw(nn2) edge [dashed, red] (nn1);
    \draw(nn1) edge [dashed, red] (nn0');
    \draw(nn0') edge node {$-1$} (s1);
    \draw(nn0') to[bend right=20] (n);


    \draw(n') edge (n1');
    \draw(n1') edge (n2');
    \draw(nn2') edge (nn1');
    \draw(nn1') edge (s0);
    
    \draw(n) edge [densely dotted, blue] (n');
    \draw(n1) edge [densely dotted, blue] (n1');
    \draw(nn2) edge [densely dotted, blue] (nn2');
    \draw(nn1) edge [densely dotted, blue] (nn1');
    \draw(n') edge (n2);
    \draw(n1') edge (e1);
    \draw(n2) edge [dashed, red] (e1);
    \draw(n2') edge (e1');
    \draw(nn2') edge (nn0');
    \draw(nn1') edge node {$-1$} (s1);
    \draw(s0) edge [loop right] (s0);
    \draw(s1) edge [loop right] (s1);


  \end{tikzpicture}
  % }
  \caption{Family of MDPs used to prove the lower bound on RPI. Red (dashed) and blue (dotted) edges from states $1, 2,  \dots, n$
  correspond to actions $0$ and $1$, respectively. Black (solid) edges from all others states are equiprobable under both actions.}
  \label{fig:mnc-mdp}
  \end{figure*}

%   \draw(n) edge node {0} (n1);
%   \draw(n1) edge node {0} (n2);
%   \draw(nn2) edge node {0} (nn1);
%   \draw(nn1) edge node {0} (nn0');
%   \draw(n) edge node{1} (n');
%   \draw(n1) edge node{1} (n1');
%   \draw(nn2) edge node{1} (nn2');
%   \draw(nn1) edge node{1} (nn1');
%   \path [->]
%   (s0)[loop, -{Latex[length=3mm,width=2mm]}] edge (s0);
