% Moved to appendix


 \begin{figure}[H]
 \centering
\scalebox{0.7}
% \scalebox{0.5}
 {
 \begin{tikzpicture}[scale=2.5]
    

\pgfarrowsdeclare{mytipnew}{mytipnew} 
{ 
  \arrowsize=0.2pt 
  \advance\arrowsize by .5\pgflinewidth 
  \pgfarrowsleftextend{-4\arrowsize-2\pgflinewidth} 
  \pgfarrowsrightextend{2\pgflinewidth} 
} 
{ 
  \arrowsize=1pt 
  \advance\arrowsize by .5\pgflinewidth 
  \pgfsetdash{}{0pt} % do not dash 
  \pgfsetroundjoin   % fix join 
  \pgfsetroundcap    % fix cap 
  \pgfpathmoveto{\pgfpoint{-4\arrowsize}{3\arrowsize}}
  \pgfpathlineto{\pgfpoint{4\arrowsize}{0\arrowsize}}
  \pgfpathlineto{\pgfpoint{-4\arrowsize}{-3\arrowsize}}
  \pgfpathlineto{\pgfpoint{0\arrowsize}{0\arrowsize}}
  \pgfpathlineto{\pgfpoint{-4\arrowsize}{3\arrowsize}}
%   \pgfusepathqstroke 
  \pgfusepathqfill
}
 \tikzset{
  % style to apply some styles to each segment of a path
  every node/.style={draw,circle,inner sep=0pt,minimum size=0pt},
  on each segment/.style={
     decorate,
     decoration={
      show path construction,
      moveto code={},
      lineto code={
         \path [#1]
         (\tikzinputsegmentfirst) -- (\tikzinputsegmentlast);
      },
      curveto code={
         \path [#1] (\tikzinputsegmentfirst)
         .. controls
         (\tikzinputsegmentsupporta) and (\tikzinputsegmentsupportb)
         ..
         (\tikzinputsegmentlast);
      },
      closepath code={
         \path [#1]
         (\tikzinputsegmentfirst) -- (\tikzinputsegmentlast);
      },
     },
  },
  mid arrow/.style={postaction={decorate,decoration={
         markings,
         mark=at position .5 with {{\arrow[#1]{mytipnew}}},
      }}},
  near arrow/.style={postaction={decorate,decoration={
         markings,
         mark=at position .4 with {{\arrow[#1]{mytipnew}}},
      }}},
  far arrow/.style={postaction={decorate,decoration={
         markings,
         mark=at position .6 with {{\arrow[#1]{mytipnew}}},
      }}},
 }


\tikzstyle{edge} = [draw,thick,postaction={on each segment={mid arrow=black}},black]
\tikzstyle{edgefar} = [draw,thick,postaction={on each segment={far arrow=black}},black]
\tikzstyle{edgenear} = [draw,thick,postaction={on each segment={near arrow=black}},black]
\newcommand{\mycircle}[1]{\large{\raisebox{.5pt}{\textcircled{\raisebox{-.9pt} {#1}}}}}
\path(0,0) node[label=below right:{\mycircle{8}}] (v0) {}
     (0,1) node (v1)[label=below left:\mycircle{7}] {}
     (1,0) node (v2) {}
     (1,1) node (v3) {}
     (0.23, 0.4) node (v4)[label=below right:\mycircle{1}] {}
     (0.23,1.4) node (v5) {}
     (1.23,0.4) node (v6) {}
     (1.23,1.4) node (v7)[label=above left:\mycircle{3}] {}
     (-1,-1) node (v8) {}
     (-1,2) node (v9)[label=above left:\mycircle{6}] {}
     (-0.66,2.7) node (v13)[label=above left:\mycircle{5}] {}
     (-0.66,-0.3) node (v12) {}
     (2,-1) node (v10)[label=below right:\mycircle{2}] {}
     (2.34,-0.3) node (v14) {}
     (2,2) node (v11)[label=below right:\mycircle{4}] {}
     (2.34,2.7) node (v15) {};
% \node[above of=v1,node distance=0.2in] (v1l) {2};
\draw[edge]
%  (v1) -- (v0)
 (v2) -- (v0)
%  (v2) -- (v3)
 (v3) -- (v1)
 (v3) -- (v11)
 (v4) -- (v0)
 (v4) -- (v6)
 (v4) -- (v12)
 (v5) -- (v1)
%  (v5) -- (v4)
 (v5) -- (v7)
%  (v5) -- (v13)
 (v6) -- (v2)
%  (v6) -- (v7)
 (v7) -- (v3)
%  (v7) -- (v15)
 (v8) -- (v0)
 (v8) -- (v9)
 (v8) -- (v10)
 (v9) -- (v1)
 (v10) -- (v2)
 (v10) -- (v11)
 (v10) -- (v14)
 (v11) -- (v9)
 (v11) -- (v15)
 (v12) -- (v8)
 (v12) -- (v13)
 (v12) -- (v14)
 (v13) -- (v9)
 (v14) -- (v6)
 (v14) -- (v15)
 (v15) -- (v13);
 \draw[edgenear] (v5) -- (v13) (v7) -- (v15) (v1) -- (v0) (v6) -- (v7);
 \draw[edgefar] (v2) -- (v3) (v5) -- (v4);
    %  \draw[edge] (v1) -- (v0);
    %  \draw[edge] (v2) -- (v0);
    %  \draw[edge] (v2) -- (v3);
    %  \draw[edge] (v3) -- (v1);
    %  \draw[edge] (v3) -- (v11);
    %  \draw[edge] (v4) -- (v0);
    %  \draw[edge] (v4) -- (v6);
    %  \draw[edge] (v4) -- (v12);
    %  \draw[edge] (v5) -- (v1);
    %  \draw[edge] (v5) -- (v4);
    %  \draw[edge] (v5) -- (v7);
    %  \draw[edge] (v5) -- (v13);
    %  \draw[edge] (v6) -- (v2);
    %  \draw[edge] (v6) -- (v7);
    %  \draw[edge] (v7) -- (v3);
    %  \draw[edge] (v7) -- (v15);
    %  \draw[edge] (v8) -- (v0);
    %  \draw[edge] (v8) -- (v9);
    %  \draw[edge] (v8) -- (v10);
    %  \draw[edge] (v9) -- (v1);
    %  \draw[edge] (v10) -- (v2);
    %  \draw[edge] (v10) -- (v11);
    %  \draw[edge] (v10) -- (v14);
    %  \draw[edge] (v11) -- (v9);
    %  \draw[edge] (v11) -- (v15);
    %  \draw[edge] (v12) -- (v8);
    %  \draw[edge] (v12) -- (v13);
    %  \draw[edge] (v12) -- (v14);
    %  \draw[edge] (v13) -- (v9);
    %  \draw[edge] (v14) -- (v6);
    %  \draw[edge] (v14) -- (v15);
    %  \draw[edge] (v15) -- (v13);
    %  \draw[selected edge] (v4) -- (v10);
    %  \draw[selected edge] (v10) -- (v7);
    %  \draw[selected edge] (v7) -- (v11);
    %  \draw[selected edge] (v11) -- (v13);
    %  \draw[selected edge] (v13) -- (v9);
    %  \draw[selected edge] (v9) -- (v1);
    %  \draw[selected edge] (v1) -- (v0);
 \end{tikzpicture}
 }
 \caption{The only 4-AUSO (up to an isomorphism) on which HPI performs $8$ vertex evaluations. The $8$ vertices are numbered in sequence. This AUSO does not satisfy the Holt-Klee conditions. Notice, for example, that the inner $3$-AUSO does not have $3$ vertex-disjoint paths from source to sink.}
 \label{fig:4auso-8hpi}
\end{figure}
